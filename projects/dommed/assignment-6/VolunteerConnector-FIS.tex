\documentclass[12pt]{article}
\usepackage{cite}
\usepackage{indentfirst}
\usepackage{subcaption}
\usepackage{graphicx}
\title{Volunteer Connector - Implementation 1}
\author{Daniel Domme (onid: dommed), \\
Charles Koll (onid: kollch), \\
Pedro Autran e Morais (onid: autranep), \\
Coulby Nguyen (onid: nguyenco), \\
Pavel Shonka (onid: shonkap)
}
\date{05 March 2018}

\begin{document}
\maketitle
\tableofcontents

\pagebreak
\section{Product Release}
\subsection{URL}
	The current state of the website can be downloaded and tried at:

	https://github.com/kollch/volunteerconnector/tree/develop

	It can also be accessed live at the following URL:

	http://ec2-18-218-108-99.us-east-2.compute.amazonaws.com/
\subsection{Setting Up}
	Functionality with the database had not been completed yet, and some basic
	functionalities like registering are awaiting this implementation in
	order for completion.

	To run the node.js web server to serve up the web page resources:
\begin{enumerate}
\item
	Node.js must be installed. Go to https://nodejs.org/ to download and
		install the software.
\item
	Open bash or putty in the volunteerconnector folder and type "npm install
		express mysql".
\item
	Type "npm start". The server should now be running on port 3000 by default.
\item
	Open up your web browser and type "http://localhost:3000/". Change the port
		according to the port that you chose to use. The website should now
		be up and running.
\end{enumerate}
\section{User Story}
	Pavel and Coulby began the Home page. Getting the home page to work correctly on
	different screen sizes has been a big issue and it is still not fixed. Also
	getting all the different parts of the website to look how we wanted has been a
	hard problem to overcome. The website is coming along slower than anticipated
	and therefore has taken more than 3 days so far. It is continuously tested
	but has yet to be fully implemented as some of the tasks have taken longer than
	expected. A major part of the website has yet to be completed. The spike diagrams
	helped to break the tasks into smaller parts and help assign them to people to
	complete. For the homepage the UML sequence has yet to be really useful as it
	is mostly just formatting the home page to look good. The UML sequence should be
	more helpful on the login page.

	Daniel, Charles, and Pedro are working on the database and where to host it.
	The major issue is that the server where we first started hosting the
	website couldn't access the information in the database due to permissions.
	Hopefully moving it to another host will fix this issue but so far that has been
	the biggest issue with the database other than getting familiar with the
	language again as well. Due to that issue it is going slower than predicted also
	and has taken more than three days as well. The diagrams we made for the
	database have been fairly sufficient so far. This part has not been tested nor
	implemented beyond just the basics so far. It has a lot left to be finished as
	just the database has been setup so far to this point. The UML diagram helps to
	figure out what subsystems we need and exactly where the data needs to be sent
	and when it needs to be received. The spike diagram wasn't really helpful as
	we didn't really split the database into smaller subprojects so the spike is
	mostly useless but the group communicates well enough that this has yet to be
	an issue.

	Daniel is working on the pop-ups, and did not have any issues with that part. The
	diagrams were helpful in showing what was needed for input boxes. This part was
	finished quickly and easily, taking only a few hours.
\section{Tests}
\subsection{Unit Test for One Existing Component}
	Unit tests were implemented for major existing components using HtmlUnit
	along with JUnit to test filling out fields in the html and testing the css. A
	Firefox browser was emulated for the unit test.

	For testing of the task of filling out the charity registration form,
	the DOM tree was traversed to find specific elements in the charity
	registration popup by selecting register, then charity registration, and
	our charity registration form was filled out using a test case. CSS selectors
	were also checked, and assertions were used to test that the correct items were
	selected and filled in.
\subsection{Unit Test for Major Existing User Story}
	The user story "Register Popup" was tested. The user story task was "There
	will be a pop up with form to input information for a new account." HtmlUnit
	was once again used to write a test case that traversed the DOM from index.html
	to get to and select the register button. Then, settings for the popup and DOM
	elements were asserted to test the state of webpage.
\subsection{Unit Test for Not Yet Implemented Part of the System}
	Currently, our team is designing our database and have yet to implement it. The
	unit tests that we are planning on conducting consist of using node.js with
	sameple data and queries written in javascript. The server will be loaded with
	information and asked to return queries which will then be tested for
	accuracy. Incorrect inputs for username and password will be tested. Along with
	this, duplications of usernames and passwords will be tested for in
	registration. The ability to add to the database and delete from it will be
	tested. Empty fields will be tested for usernames and passwords. The use
	of characters and long strings and text will be tested for in saving to the
	database and then queried for.
\section{Design changes and rationale}
	No questions were asked of the customer. All tasks seemed manageable and
	clear to the development team.

	There have been no major changes in the design of the website yet. There have
	been issues with hosting and connecting the database to the website. Rehosting
	of the database is nearing completion, and it seems as if connectivity
	between the web server and the database is working. One thing that was not
	implemented was the error pop up for a nonexistent user. This feature will be
	implemented early next week on 06 March 2018.
\section{Meeting Report}
\subsection{Plan for Next Week}
	Next week, our team needs to complete the following tasks: user not found
	error pop up, interface between the database and the website, functionality of
	charities to post events, ability to search and filter event posts,
	functionality to select a single event post, ability for a volunteer to click
	on the apply button and let the charity know they are interested in
	helping, search events using distance slider to find posts near the user's
	location functionality, showing the volunteer their current event sign-ups,
	having recent posts show up first on the homepage, a homepage button, an about
	page for information on the website and how to use it, a list of older posts
	displayed on the homepage after the most recent posts, and organizational page
	for charities. We plan on finishing up the remainder of the site next week.
\subsection{What We Accomplished this Week}
	This week, the development team completed the register and login pop-ups and
	some of the home page functionality. Most of the user stories were completed on
	time. The node.js web server was also completed, and newest posts are displayed
	at the top in little boxes. We have started implementing the database by
	fleshing out the scheme and implementing this into tables. The database server has
	also been connected to the web server, but interface between the two servers
	still needs to be completed.
\subsection{Customer Interaction}
	The customer was willing to meet with the development team this week, and they
	were able to help with design questions. The customer was also very reasonable in
	assigning tasks to be completed.

\quad

\subsection{Contributions}
\begin{itemize}
\item
	Daniel Domme: Charity and volunteer registration pop up forms, sign-in form, unit
		tests, program instructions
\item
	Charles Koll: Implementation of the database, server hosting, Latex document compilation
\item
	Pedro Morais: Worked on server, hosting on AWS, readme
\item
	Coulby Nguyen: Worked on creating user stories, and implemented two of
		the stories, including the homepage and the user interface for a
		charity to add information to create a post.
\item
	Pavel Shonka: Worked on home page, wrote up user stories
\end{itemize}
\end{document}
