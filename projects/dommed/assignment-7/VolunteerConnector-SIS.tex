\documentclass[12pt]{article}
\usepackage{cite}
\usepackage{indentfirst}
\usepackage{subcaption}
\usepackage{listings}
\usepackage{graphicx}
\title{Volunteer Connector - Implementation 2}
\author{Daniel Domme (onid: dommed), \\
Charles Koll (onid: kollch), \\
Pedro Autran e Morais (onid: autranep), \\
Coulby Nguyen (onid: nguyenco), \\
Pavel Shonka (onid: shonkap)
}
\date{12 March 2018}

\begin{document}
\maketitle
\tableofcontents

\pagebreak
\section{Product Release}
	The website can currently be accessed live at:

	http://web.engr.oregonstate.edu/~kollch/volunteerconnector/

	\quad

	The code and instructions for hosting the website can be found at:

	https://github.com/kollch/volunteerconnector
\section{User Story}
\begin{itemize}
\item
	The Database Interface user stary was implemented by both Daniel and Charles.
	Initially the interface was going to be using the MySQL module with a
	Node.Js server, but complications arose on implementation and
	we migrated the server code from Node.Js to PHP. The two worked together
	using paired programming at the same time to come to a successful
	solution. The user story was implemented and tested. The class diagram
	and database schema diagram were both useful documents in completing the server.
\item
	The Home Button user story has been completed by Daniel. Pairs were not used to
	implement this story. The Home button is implemented and tested. The
	task took one day to complete. One problem that was encountered was the
	transition from Node.Js to PHP. The file extensions for the CSS and
	Javascript had to be changed. The spike was useful in completing this
	task, and the mockups were also helpful in making this page.
\item
	The Event info user story was implemented by Coulby and Charles. The layout
	and design of the posts was completed by Coulby. The querying of the database was
	done by Charles. A problem that was run into was fields in the post not matching
	the table in the database. This was fixed by editing the post to match the
	database. The implementation took one week to complete. The user story was
	helpful in helping us to know what needs to be done. Also, the class diagram was
	helpful in determining what was going to be in the database and the webpage.
\item
	The Filter query user story was implemented by Pedro. Pairs were not used for
	this story. The task required 2 days to complete. The task was implemented and
	tested. There were no problemn in implementing this into the website. While not
	a fixed search bar with advanced search, the current state of the filter query
	passes the requirements.
\item
	The Looking at 1 specific post user story was completed by Coulby. The story
	still needs to be completely incorporated into the website when a user clicks
	on a post. The page also needs to pull data from the database. This
	feature will continue to be worked on and so far has taken two days to
	complete. A problem that we have run into is being able to query the
	database properly. No documents were of particular use for this task. Some
	work by Charles may be needed to complete this.
\item
The Post information (searchable keywords) user story was partially completed
	by Pedro. The search looks at the posts in the HTML and finds matching
	information. Only matching posts are displayed. The functionality needs
	to be extended into the database to search all posts available. This
	user story has taken two days so far and will be continually worked
	on until complete. The mockups document was very helpful for this
	user story.
\item
	The Volunteer Button user story has not been completed. This feature is
	considered a convenience function by the customer and is of lower
	priority than outstanding features as of now. If time affords completion
	of this feature, the team might be able to work on it next week.
\item
	The Search distance user story has not been completed. While the customer
	considers this feature to be important, the implementation of
	other incomplete features has higher priority. Authorization has
	been obtained to delay the implementation of this until the database
	is completely hooked up.
\item
	The Looking at user events user story is on the same level of priority and status
	as the volunteer button user story. The customer is willing to wait for
	this feature to be completed. This feature is also dependent on the
	volunteer button.
\item
	The Recent posts (appear on front page) user story was implemented by
	Coulby. This task took one week, but it is not completely finished. The
	recent posts still need to be pulled from the database and then rendered.
	There is currently the problem of creating new DOM elements with imported data.
	The most recent posts are placeholders for the final version of the site. The
	mockups that were made for this site were very helpful in the completion of this
	feature.
\item
	The About user story was completed by Coulby. This task took two days. There
	were no issues in creating this feature. This task has been implemented and
	tested. The user stories were helpful in making this task on time and to
	requirements.
\item
	The Other posts user story has been changed by the customer to be similar to the
	recent posts user story. All displayed posts will not be in the same format.
	The amount of displayed posts still needs to be decided on.
\item
	The Login popup user story was completed by Daniel. The task took two days. The
	only thing left to do is create a session in PHP and actually log the user in.
	This will be done by Charles and Daniel. There have been no major issues in
	doing this feature, and the feature is implemented.
\item
	The Organization info user story still needs to be completed. The customer has
	agreed to put this feature on hold until other features have been completed.
	This feature is not a necessity to run. Until this feature is implemented, users
	will be linked to the Charities' websites.
\end{itemize}
\section{Design Changes and Rationale}
\subsection{Customer Questions}
	Questions asked to the customer revolved around the requirements and
	timetables of the project. The customer was available for questioning and was
	very helpful. Some questions were:
\begin{itemize}
\item
	Are you willing to put off the implementation of the charity organization page,
	distance search feature, the volunteer button, and the user volunteer display?
	\begin{itemize}
	\item
		The customer said yes to all
	\end{itemize}
\item
	Our team thinks that it might be better to merge the older posts into the
	recent post format. What are your thoughts?
	\begin{itemize}
	\item
		The customer said that they think that it is a good idea. This will
		look better and be less confusing.
	\end{itemize}
\end{itemize}
\subsection{Design Changes}
	The server for the website was originally writted in Node.Js. We had trouble
	connecting Node.Js to our database, which is being developed by Charles, as he
	does not know Node.Js very well which made it hard for us to troubleshoot the
	server code with the database. To make it so the server team and the database
	team could collaborate more easily, we switched the backend specification to
	use PHP instead. Switching to PHP also made it so we could serve the website
	directly on the OSU Flip server, which negated the need to host on Amazon Web
	Services.

	We also changed the design of the navigation bar, because the original nav bar
	made it hard to format the register, login and search bars. The new navigation
	bar includes a home button, which was one of our user stories, and also links to
	an account and an about page. This was a natural way to incorporate our
	user account user story. Our "about" design and user story originally had an
	about section on the main landing page, but we decided that with the navigation
	bar it was more natural to give it its own subpage and link, which also freed up
	space on the front page which we believe will lead to a better user interface and
	execution for other user stories, like the "recent posts" and "other posts" user
	stories.

	Another change was the merge of the older post display format into the same
	format as the recent posts. This reason for this was partially time and partially
	design. The recent post looks far better. This was done with the consent of the
	customer.

	A few requirement changes were the implementation of the volunteer button,
	the user event info, organization info page, and the search distance were all
	lowered in priority and the implementation has been put off until more
	important features are completed. This was done after the customer was spoken
	to and a reappraisal was completed. Ultimately, the customer agreed to the
	changes.
\subsection{User Story Changes}
\begin{itemize}
\item
	New Story: implement database that the website pulls data from. Task time: 1
	week. Priority: top.
\item
	Older Story Priority Change: volunteer button. Task time: 3 days. New Priority:
	low/bottom
\item
	Older Story Priority Change: organization info page. Task time: 3 days. New
	Priority: low/bottom
\item
	Older Story Priority Change: search distance. Task time: 3 days. New Priority:
	low/bottom
\item
	Older Story Priority Change: volunteer's events. Task time: 3 days. New Priority:
	low/bottom
\end{itemize}
\section{Refactoring}
	During the programming of this project, the team did several refactors. One
	major one was changing from tabs to two spaces for the indentation of the
	code. Previously, the code was a mess and hard to read. A second refactoring
	was to make different support files for pages. For example: index.php has
	style.css and index.js, but account.php has style2.css and account.js. This
	made it easier to know which code affected what and made the pages more modular.
\section{Tests}
	During integration of new features into the existing code, we tested our code
	using some basic unit tests. With the help of PHPUnit and the Eclipse IDE, inputs
	for fields in forms for adding to the database were tested. Abnormally long
	strings of characters were tested along with special characters. In
	reality, we do not expect the users to enter anything unusual into the code. We
	did not have any major errors. One place that we did notice a slight bug was
	submitting half-filled forms. We found out that we needed to not allow the user
	to submit without a complete form filled out otherwise there were errors in
	searching the database. We will implement checks in either Javascript or the
	PHP query of the SQL database to find and reprompt the user.

	For testing the database, we simply made a simple Javascript program using
	Node.js with many data queries adding to and requesting from the database. The
	strings to input into the database were formed using random string generators
	of varying lengths. Then we output the status of the queries and the data in the
	database to test the functionality. Errors output threw exceptions. One
	input is as follows:
\begin{lstlisting}
var sql = "INSERT INTO user (username, password)
           VALUES ('Ted Kazynkski', 'airforce1')";
con.query(sql, function (err, result) {
  if (err) throw err;
    console.log("1 record inserted");
});
\end{lstlisting}

	Another test on the database was searching the tables for entries. After
	randomly populating the tables, random strings were used to search the database
	along with the same strings that populated the database. The output of the
	queries and errors were printed to the console for the user to discern. An example
	of a search of the database is:
\begin{lstlisting}
con.query("SELECT * from user WHERE username = 'Francis221'",
function (err, result) {
  if (err) throw err;
    console.log(result);
});
\end{lstlisting}

	These tests were run on most parts of the database schema. There were no major
	issues found.
\section{Meeting Report}
	Next week, our group plans to finish all features of the website. This
	includes the addition of the user stories for the volunteer button, charity
	info page, volunteer events saved, and the distance search. In addition, the full
	implementation of features such as pulling events from the database will be
	done. This should take no more than a week to finish up.

	This week, the team finished the about page, hooked up the database to the
	site, implemented a search feature, and made individual event pages. Also, a
	slideshow will be completed for the team presentation of our software. We also
	spoke with the customer on delaying several features until next week.

	During this part of the project, the customer was willing to meet with the team
	and make changes to the requirements. The customer was very reasonable and
	forgiving for the backlog of features.

\quad

\subsection{Contributions}
\begin{itemize}
\item
	Daniel Domme: Worked on the defunct Node.Js, Javascript and HTML for the
	homepage and account page, and assignment 7 writeup.
\item
	Charles Koll: LaTeX document compilation, database implementation, PHP
	(queries database and passes results to Javascript)
\item
	Pedro Morais: Design changes, user stories, search bar 
\item
	Coulby Nguyen: Worked on what the home page, about page, account page, and
	the single post page should look like, then wrote the Javascript function
	code for dynamically creating the page so the event posts are based on
	information from the database.
\item
	Pavel Shonka: Example posts and part of slide show
\end{itemize}
\end{document}
